\documentclass[11pt]{article}
\usepackage{geometry}
\geometry{letterpaper}
\usepackage{graphicx}
\usepackage{caption}
\usepackage{subcaption}
\usepackage{amssymb}
\usepackage{hyperref}
\usepackage{enumitem}
\usepackage[utf8]{inputenc}
\newcommand{\Mod}[1]{\ (\mathrm{mod}\ #1)}
\usepackage{amsmath}

\title{MAT315 - HW3}
\author{Quentin Vilchez -- 1002562586}

\begin{document}
\maketitle
\begin{enumerate}
\item{
\begin{enumerate}
\item $3x \equiv 1 \Mod{11}$, we know that $(11,3) = 1$\\
This equation is soluble.\\
$\exists x,y \in Z$ such that $3x -11y =1$.\\$(4,1)$ are solutions. Therefore $3x \equiv 1 \Mod{11} \iff x \equiv 4 \Mod{11}$.
\item $2x \equiv 1 \Mod{11}$, we know that $(11,2) = 1$\\
This equation is soluble.\\
$\exists x,y \in Z$ such that $2x -11y =1$.\\$(6,1)$ are solutions. Therefore $2x \equiv 1 \Mod{11} \iff x \equiv 6 \Mod{11}$.
\item $37x \equiv 2 \Mod{145}$, we know that $(145,35) = 1$\\
This equation is soluble.\\
$\exists x,y \in Z$ such that $37x -145y =2$.\\$(-94,-24)$ are solutions. Therefore $37x \equiv 2 \Mod{145} \iff x \equiv 51 \Mod{145}$.
\item $15x \equiv 5 \Mod{305}$, we know that $(305,15) = 5$\\
This equation is soluble.\\
$\exists x,y \in Z$ such that $15x -305y =5$.\\$(-20,-1)$ are solutions. Therefore $15x \equiv 5 \Mod{305} \iff x \equiv 285 \Mod{305}$.
\item $18x \equiv 6 \Mod{45}$, we know that $(45,18) = 9$\\
This equation is not soluble.\\
\end{enumerate}}
\item{\begin{enumerate}
\item $$ x \equiv 0,1,2,3,4,5,6,7,8,9,10,11,12 \Mod{13}$$
Then,
$$x^2 \equiv 0,1,4,9,3,12, 10,10, 12, 3, 9, 4, 1 \Mod{13}$$
So the residue classes of $x^2 \Mod{13}$ are $0,1,4,9,3,12, 10$.
\item $2x^2 \equiv 1 \Mod{13}$, we have $(13, 2) = 1$, so the equation seems soluble.\\
$(-6, -1)$ is a solution. $2x^2 \equiv 1 \Mod{13} \iff x^2 \equiv 7 \Mod{13}$. But in (a) we saw that $x^2 \not\equiv 7 \Mod{13}$. Therefore the equation is not soluble. 
\item Suppose there exists $x,y \in Z$ such that $13x^3-11y^2 =1$. Then we must have,
\begin{equation} 
\begin{split}
-11y^2 & \equiv 1 \Mod{13} \\
11y^2 & \equiv 12 \Mod{13}
\end{split}
\end{equation}
Now $(12,13) =1$. Therefore, $\exists k,l \in Z$ such that $11k - 13l  =12$. 
\\$(72,60)$ is a solution. Hence, $11y^2  \equiv 12 \Mod{13} \iff y^2 \equiv 7 \Mod{13}$. But in (a) we saw that $y^2 \not\equiv 7 \Mod{13}$. Therefore this equation has no solutions in $\mathbb{Z}$.
\item It is easy (but tedious) to check that the residue classes for $x^3 \Mod{11}$ are $0,1,2,3,4,5,6,7,8,9,10$. 
$$13x^3-11y^2\equiv 1 \Mod{11} \iff 13x^3 \equiv 1 \Mod{11}$$
This equation is soluble since $(13,11)= 1$ and the residue classes for $x^3 \Mod{13}$ are the same as $x \Mod{13}$. 
\end{enumerate}}
\item{\begin{enumerate}
\item $$x^2 \equiv 1 \Mod{p} \Leftrightarrow (x-1)(x+1) \equiv 0 \Mod{p}$$
$$\Leftrightarrow p \mid (x-1)(x+1) \Rightarrow p \mid (x-1) \mbox{ or } (x+1)$$
\begin{align*}
x-1 &\equiv 0 \Mod{p}           &  x+1 &\equiv 0 \Mod{p} \\
x &\equiv 1 \Mod{p}    &  x &\equiv -1 \Mod{p}   
\end{align*}
\item This is true.
{\begin{enumerate}
\item \textit{Existence:} \\We have $a \equiv a \Mod{p}$ and $(a,p) =1$. \\
Therefore there exists $x$ and $y$ such that $ax -py= 1 \Rightarrow ax \equiv 1 \Mod{p}$. \\ We may simply take $b = x \Mod{p}$.
\item \textit{Uniqueness:} \\ Suppose there exists $1\leq b_1, b_2 \leq p-1$ such that $ab_i \equiv 1 \Mod{p}$ for $i = 1, 2$.
\\Then, $a(b_1-b_2) \equiv 0 \Mod{p}$. So $p\mid (b_1 -b_2)$ hence $b_1\equiv b_2 \Mod{13} \Rightarrow b_1 =b_2$. 
\end{enumerate}}
\item We will first show that $(p-2)! \equiv 1 \Mod{p}$.\\
We first note that there is an even number of terms in the product $(p-2)!$ (since we may neglect the term 1). 
\\By (a), we know that $x^2 \equiv 1 \Mod{p} \Leftrightarrow x \equiv \pm 1 \Mod{p}$, therefore (by (b)) for each term in the product , we can find a unique term (not itself) such that $ab \equiv 1\Mod{p}$, adding this to the fact that there are an even number of terms in the product $(p-2)!$, we get $(p-2)! \equiv 1 \Mod{p}$.
\\ Now $(p-2)!(p-1) \equiv p-1 \Mod{p} \equiv -1 \Mod{p}$. 
\item{\begin{enumerate}


\item  Let $R = \lbrace r_1, r_2, \cdots , r_{\phi(p^c)}\rbrace$ be a complete set of residues prime to $p_c$.
Then for each $r_i$ there exists a unique $r_j$ such that $r_jr_i \equiv 1 \Mod{p^c}$ since $(r_i, p^c) =1$.
\item Now for $x\in R$, $x^2 \equiv 1 \Mod{p^c} \Leftrightarrow p^c \mid (x+1)(x-1) \Leftrightarrow x = 1 \mbox{ or } p^c-1 \Leftrightarrow x\equiv \pm 1 \Mod{p^c}$. 
\item So now we consider $K = r_1r_2\cdots r_{\phi(p^c)}$, where $r_1 = 1$ and  $r_{\phi(p^c)} = p^c-1$. It is easy to see that $K' =r_2\cdots r_{\phi(p^c)-1} \equiv 1\Mod{p^c}$ since $K'$ has an even number of terms ($\phi(p^c) =p^c-p^{c-1}$ which is even) and by (ii). Therefore $K \equiv r_{\phi(p^c)} \Mod{p^c} \equiv -1 \Mod{p^c}$.
\end{enumerate}}
\item A complete set of residues prime to 15 is $\lbrace 1, 2, 4, 7, 8, 11, 13, 14\rbrace$\\
$1 \times 2 \times 4\times 7 \times 8\times 11 \times 13 \times 14 = 896896$
and $$192192 \equiv 1 \Mod{15}$$
\end{enumerate}}
\item{\begin{enumerate}
\item $\phi(n) = \frac{1}{3}n \Leftrightarrow n = 2^{c_1}3^{c_2}$ where $c_i \geq 1$.
Indeed, $$\phi(n) = n \prod_{p \mid n} (1-\frac{1}{p}) = n \times \frac{1}{2} \times \frac{2}{3} = \frac{1}{3}n$$
\item $\phi(n) = \frac{1}{24} n $ is not possible. 
\\ Write $$\phi(n) = n \prod_{p \mid n} (1-\frac{1}{p}) =n \times \frac{(p_1 -1)(p_2-1) \cdots (p_k-1)}{p_1\cdots p_k}$$
where $p_i$ are the prime divisors of $n$.\\
Let $ A = \frac{(p_1 -1)(p_2-1) \cdots (p_k-1)}{p_1\cdots p_k}$\\
Now we know that all prime numbers greater than 2 are odd. \\So if $n$ is odd our numerator in $A$ cannot be 1 since $(p_i-1, p_1\cdots p_k) = 1$ for all $i= 1, \cdots k$ (except in the case of (a)). \\If $n$ is even, then our numerator can be written as $2k$ where $k\geq 1 $ and even, and the denominator can be written as $2l$ with $l\geq 1$ and odd. So we would get $A = \frac{2k}{2l} = \frac{k}{l} \not = \frac{1}{24}$.
 
\item $\phi(2n) = \phi(n) \Leftrightarrow n \mbox{ is odd}$. 
Take $n$ odd, $$ \phi(2n) = 2n \prod_{i} \frac{1}{2} \times(1-\frac{1}{p_i}) = n \prod_{p \mid n} (1-\frac{1}{p}) = \phi(n)$$
where $p_i$ are the prime divisors of $n$.
\end{enumerate}}
\item{\begin{enumerate}
\item Suppose f is multiplicative. Consider $n_1,n_2 \in \mathbb{Z}$ such that $(n_1,n_2)= 1$.\\
If $d\mid n_1 n_2$, then $d$ can be uniquely written as $d = k_1 k_2$ whenr $k_i \mid n_i$, since $n_1$ and $n_2$ are coprime. 
\begin{equation} 
\begin{split}
g(n_1n_2) & = \sum_{d\mid n_1n_2} f(d) = \sum_{k_1\mid n_1, k_2\mid n_2} f(k_1k_2)\\
&=\sum_{k_1\mid n_1, k_2\mid n_2} f(k_1)f(k_2)=  \sum_{k_1\mid n_1} f(k_1)\sum_{k_2\mid n_2} f(k_2) = g(n_1)g(n_2)
\end{split}
\end{equation}
\item We know that the identity function is multiplicative. Therefore, by (a) $\sigma (n) = \sum_{d\mid n} d$ is multiplicative.
\item The divisors of $p^c$ are $\lbrace 1, 2, \cdots , p^{c-1}, p^c\rbrace.$
So, $$ \sigma(p^c) = \sum_{d\mid p^c} d = \sum_{n=0}^{c} p^n$$
\item Let $p_{1}^{c_1} \cdots p_{k}^{c_k}$ be the prime decomposition of n. We get, 
$$ \sigma(n) = \sigma(p_{1}^{c_1} \cdots p_{k}^{c_k}) = \sigma(p_{1}^{c_1}) \cdots \sigma( p_{k}^{c_k})= \prod_{i=1}^{k}  \bigg[\sum_{n=0}^{c_i} p_i^n \bigg] $$

\end{enumerate}}
\item{\begin{enumerate}
\item If $x_0, x_1$ both are solutions, then $x_0 \equiv x_1 \Mod{n_i}$ for all $i$. We know that $(n_i,n_j) =1 $ for $i \not = j$. Hence, by theorem 53 of Hardy-Wright, $$x_0 \equiv x_1 \Mod{n_i n_j}$$ and since $(\prod_{i \not = j} n_i, n_j) = 1$, we have $$x_0 \equiv x_1 \Mod{N}$$
\item{\begin{enumerate}
\item $N_ix_i \equiv ci \Mod{n_i}$, $(N_i, n_i) = 1$, therefore this equation is soluble and has a unique solution $\Mod{n_i}$. 
\item $N_j x_j \equiv 0 \Mod{n_i}$ since $N_j = \prod_{i \not = j} n_i$, so $\forall i\not =j $ $n_i\mid N_j$.
\item $x = \sum N_ix_i$, by (i) there exists $x_i$ such that $N_ix_i \equiv ci \Mod{n_i}$, and by (ii) we have $$x \equiv N_ix_i \Mod{n_i} \equiv ci \Mod{n_i}$$
So $x$ is a solution to the system of congruences.

\end{enumerate}}
\item $$ \begin{cases} x \equiv 3 \Mod{4} \\ x \equiv 2 \Mod{3}\\ x \equiv 1 \Mod{5} \end{cases}$$
Define $N_1 = 4\times 3$, $N_2 = 4\times 5$, $N_3 = 3\times 5$.
We get $$ \begin{cases} N_1 x_1 \equiv 1 \Mod{5} \Leftrightarrow x_1 \equiv 3\Mod{5} \\ N_2 x_2 \equiv 2 \Mod{3} \Leftrightarrow x_2 \equiv 1 \Mod{3} \\ N_3x_3 \equiv 3 \Mod{4} \Leftrightarrow x_3 \equiv 1 \Mod{4} \end{cases}$$
So $x = 20 \times 1 + 15\times 1 + 12 \times 3 = 71$ is a solution. 
\end{enumerate}}
\end{enumerate}


\end{document}